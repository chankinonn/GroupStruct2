\nonstopmode{}
\documentclass[a4paper]{book}
\usepackage[times,inconsolata,hyper]{Rd}
\usepackage{makeidx}
\usepackage[utf8]{inputenc} % @SET ENCODING@
% \usepackage{graphicx} % @USE GRAPHICX@
\makeindex{}
\begin{document}
\chapter*{}
\begin{center}
{\textbf{\huge Package `GroupStruct2'}}
\par\bigskip{\large \today}
\end{center}
\inputencoding{utf8}
\ifthenelse{\boolean{Rd@use@hyper}}{\hypersetup{pdftitle = {GroupStruct2: Interactive, User-Friendly Shiny Application for Statistical Analysis and Data Visualization in Species Diagnosis}}}{}
\ifthenelse{\boolean{Rd@use@hyper}}{\hypersetup{pdfauthor = {Kin Onn Chan}}}{}
\begin{description}
\raggedright{}
\item[Type]\AsIs{Package}
\item[Title]\AsIs{Interactive, User-Friendly Shiny Application for Statistical
Analysis and Data Visualization in Species Diagnosis}
\item[Version]\AsIs{1.0.0}
\item[Author]\AsIs{Kin Onn Chan}
\item[Maintainer]\AsIs{Kin Onn Chan }\email{chankinonn@users.noreply.github.com}\AsIs{}
\item[Description]\AsIs{>
GroupStruct2 is an interactive Shiny application designed for 
analyzing morphological, meristic, and mixed species data. 
It provides modules for data input, summary statistics, 
allometric corrections, multivariate analyses (e.g., PCA, DAPC, MFA), 
and rich data visualizations to assist in species delimitation 
and comparative morphology studies.}
\item[License]\AsIs{MIT + file LICENSE}
\item[Encoding]\AsIs{UTF-8}
\item[LazyData]\AsIs{true}
\item[Imports]\AsIs{shiny, DT, dplyr, ggplot2, tidyr, vegan, viridis,
RColorBrewer, rstatix, car, readr, adegenet, FactoMineR,
factoextra, shinyjs, colourpicker, forcats, purrr, scales,
openxlsx, shinyWidgets, ggthemes, broom, tibble, htmltools,
stringr, ggpubr, ggrepel}
\item[Remotes]\AsIs{arleyc/PCAtest}
\item[RoxygenNote]\AsIs{7.2.3}
\item[Suggests]\AsIs{testthat (>= 3.0.0)}
\item[Config/testthat/edition]\AsIs{3}
\end{description}
\Rdcontents{\R{} topics documented:}
\inputencoding{utf8}
\HeaderA{groupstruct2}{Launch the GroupStruct2 Shiny app}{groupstruct2}
%
\begin{Description}
This function launches the Shiny app from the package.
\end{Description}
%
\begin{Usage}
\begin{verbatim}
groupstruct2()
\end{verbatim}
\end{Usage}
\inputencoding{utf8}
\HeaderA{mod\_home\_ui\_combined}{Home UI for Combined Data}{mod.Rul.home.Rul.ui.Rul.combined}
%
\begin{Description}
Home UI for Combined Data
\end{Description}
%
\begin{Usage}
\begin{verbatim}
mod_home_ui_combined(id)
\end{verbatim}
\end{Usage}
%
\begin{Arguments}
\begin{ldescription}
\item[\code{id}] Namespace ID
\end{ldescription}
\end{Arguments}
%
\begin{Value}
Shiny UI for combined/mixed data home screen
\end{Value}
\inputencoding{utf8}
\HeaderA{mod\_home\_ui\_meristic}{Home UI for Meristic Data}{mod.Rul.home.Rul.ui.Rul.meristic}
%
\begin{Description}
Home UI for Meristic Data
\end{Description}
%
\begin{Usage}
\begin{verbatim}
mod_home_ui_meristic(id)
\end{verbatim}
\end{Usage}
%
\begin{Arguments}
\begin{ldescription}
\item[\code{id}] Namespace ID
\end{ldescription}
\end{Arguments}
%
\begin{Value}
Shiny UI for meristic home screen
\end{Value}
\inputencoding{utf8}
\HeaderA{mod\_home\_ui\_morphometric}{Home UI for Morphometric Data}{mod.Rul.home.Rul.ui.Rul.morphometric}
%
\begin{Description}
Home UI for Morphometric Data
\end{Description}
%
\begin{Usage}
\begin{verbatim}
mod_home_ui_morphometric(id)
\end{verbatim}
\end{Usage}
%
\begin{Arguments}
\begin{ldescription}
\item[\code{id}] Namespace ID
\end{ldescription}
\end{Arguments}
%
\begin{Value}
Shiny UI for morphometric home screen
\end{Value}
\printindex{}
\end{document}
